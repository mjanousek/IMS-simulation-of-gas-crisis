\documentclass[a4paper,11pt]{article}

\usepackage[czech]{babel}
\usepackage[utf8]{inputenc}
\usepackage{graphics}
\usepackage[hidelinks]{hyperref}
%velikost stránky
\usepackage[left=1.5cm,text={18cm, 25cm},top=2.5cm]{geometry}


%%%%%%%%%%%%%%%%%%%%%%%%%%%%%%%%%%%%%%%%%%%%%%%%%%%%%%%%%%%%%%%%%%%%%%%%%%%%%
\begin{document}
\begin{center}
\Huge
\textsc{\LARGE Vysoké učení technické v Brně\\[-4.5mm] Fakulta informačních technologií}\\
\vspace{\stretch{0.100}}
\includegraphics{logo_fit.png}\\
\vspace{\stretch{0.100}}
{\LARGE Modelování a simulace\\[0mm]2014/2015}\\[2mm]
Plynová krize v Evropě\\
\vspace{\stretch{0.618}}
\end{center}
{\Large 
Marek Fiala, xfiala46\\
Martin Janoušek, xjanou14
\hfill
\today 
}

\thispagestyle{empty}
\newpage

\thispagestyle{empty}
\tableofcontents
\newpage


\section{Úvod} 

Úvod musí vysvětlit, proč se celá práce dělá a proč má uživatel výsledků váš dokument číst (prosím, projekt sice děláte pro získání zápočtu v IMS, ale mohou existovat i jiné důvody). Případně, co se při čtení dozví.

Například: 
v této práci je řešena implementace ..., která bude použita pro sestavení modelu ...
na základě modelu a simulačních experimentů bude ukázáno chování systému ... v podmínkách ...
smyslem experimentů je demonstrovat, že pokud by ..., pak by ...
Poznámka: u vyžádaných zpráv se může uvést, že zpráva vznikla na základě požadavku ... (u školní práce takto zdůvod'novat projekt ovšem nelze, že). Je velmi praktické zdůvodnit, v čem je práce náročná a proto přínos autora nepopiratelný (např. pro zpracování modelu bylo nutno nastudovat ..., zpracovat, ... model je ve svém oboru zajímavý/ojedinělý v ...).


\vline

Tato práce popisuje řešení implementace diskrétního modelu produkce, spotřeby a přepravy plynu v zemích Česká republika, Polsko, Slovensko, Maďarsko a Ukrajina.

Na základě vytvořeného modelu a simulačních experimentů bude ukázána závislost jednotlivých zemí na dovozu zemního plynu z Ruska, 
užitečnost možného dovážení zkapalněného zemního plynu \textit{(LNG)} z USA a míra připravenosti výše zmiňovaných zemí na možnou plynou krizi.

Smyslem experimentů je demonstrovat odolnost daných států vůči embargu na zemní plyn ze strany Ruska a jejich schopnost udržet normální běh státu v tomto stavu.


\subsection{Autoři a hlavní zdroje informací}
\subsubsection{Autoři}
Podkapitoly:
kapitola 1.1: Kdo se na práci podílel jako autor, odborný konzultant, dodavatel odborných faktů, význačné zdroje literatury/fakt, ...
je ideální, pokud jste vaši koncepci konzultovali s nějakou autoritou v oboru (v IMS projektu to je hodnoceno, ovšem není vyžadováno)
pokud nebudete mít odborného konzultanta, nevadí. Nelze ovšem tvrdit, že jste celé dílo vymysleli s nulovou interakcí s okolím a literaturou.


\begin{itemize}
\item Marek Fiala, xfiala46@stud.fit.vutbr.cz
\item Martin Janoušek, xjanou14@stud.fit.vutbr.cz
\end{itemize}

\subsubsection{Hlavní zdroje informací}
Za hlavní zdroj informací byl použit elektronický sborník Natural Gas Information \cite{IEA}\footnote{konkrétně ten vydaný v roce 2014},
který je každoročně vydáván \textit{Mezinárodní energetickou agenturou}\footnote{viz. http://www.iea.org/}.
Tento elektronický sborník se zabívá požadavky a zásobami plynu v zemích po celém světě.


Dalším důležitým zdrojem informací byla webová stránka asociace \textit{Gas Infrastructure Europe (GIE)}\footnote{viz. http://www.gie.eu.com/}.

\subsection{Validace modelu}

kapitola 1.2: V jakém prostředí a za jakých podmínek probíhalo experimentální ověřování validity modelu – pokud čtenář/zadavatel vaší zprávy neuvěří ve validitu vašeho modelu, obvykle vaši práci odmítne už v tomto okamžiku.

\subsection{Zadání projektu}

Oficiální zadání projektu je k dispozici na stránkách \url{http://perchta.fit.vutbr.cz:8000/vyuka-ims/34}.

\begin{quotation}
\uv{Navrhněte diskrétní model produkce, spotřeby a přepravy plynu v zemích okolo České republiky (Polsko, ČR, Slovensko, Maďarsko, Ukrajina).
Za místa produkce považujte místa aktuálních (a uvažovaných) přístavních terminálů na LNG v Evropě, podzemní zásobníky v daných zemí (včetně jejich kapacity) a Rusko.
Každou zemi považujte v modelu za jeden uzel produkce a spotřeby (neuvažují se místní rozvody plynu). Modelujte současnou přepravní síť mezi uzly.
Síť dle potřeb rozšiřte o další uzly a země. Navrhněte a zdůvodněte přepravní doby mezi uzly v jednotce množství za hodinu
(př. uzel Rusko produkuje do přepravního kanálu na Ukrajinu X množství každou hodinu. Přepravní doba z uzlu Rusko do uzlu Ukrajina je Y hodin.).
Zkoumejte dostupnost plynu v uzlech v různých obdobích roku (zimní spotřeba, letní spotřeba) v různých scénářích (zprovoznění terminálu LNG pro dovoz z USA,
Ukrajina přeruší transportní cestu, Rusko přestane dodávat apod.).}
\end{quotation}


\section{Rozbor tématu a použitých metod/technologií}

Shrnutí všech podstatných faktů, které se týkají zkoumaného systému (co možná nejvěcnějším a technickým (ideálně formálně matematickým) přístupem, žádné literární příběhy). Podstatná fakta o systému musí být zdůvodněna a podepřena důvěryhodným zdrojem (vědecký článek, kniha, osobní měření a zjišťování). Pokud budete tvrdit, že ovce na pastvě sežere dvě kila trávy za den, musí existovat jiný (důvěryhodný) zdroj, který to potvrdí. Toto shrnutí určuje míru důvěryhodnosti vašeho modelu (nikdo nechce výsledky z nedůvěryhodného modelu). Pokud nebudou uvedeny zdroje faktů o vašem systému, hrozí ztráta bodů.



\subsection{Popis použitých metod a technologii}

kapitola 2.1: Popis použitých postupů pro vytvoření modelu a zdůvodnění, proč jsou pro zadaný problém vhodné. Zdůvodnění může být podpořeno ukázáním alternativního přístupu a srovnáním s tím vaším. Čtenář musí mít jistotu, že zvolené nástroje/postupy/technologie jsou ideální pro řešení zadaného problému (ovšem, "dělám to v Javě, protože momentálně Java frčí..." nemusí zadavatele studie uspokojit).

\vline

Pro implementaci byl zvolen jazyk C++ a to ve variantě bez knihoven podporujících simulaci.
Důvodem tohoto rozhodnutí byla rychlost spustitelného programu a také to,
že veškeré události v modelu se vykonávají diskrétně v intervalech po jedné hodině.
\textbf{Modelového času} pro účeli tohoto modelu lze tedy docílit jednoduše vykonáním všech
událostí se stejným počátečním časem v jedné iteraci cyklu.


\vline

kapitola 2.2: Popis původu použitých metod/technologií (odkud byly získány (odkazy), zda-li jsou vytvořeny autorem) - převzaté části dokumentovat (specificky, pokud k nim přísluší nějaké autorské oprávnění/licence). Zdůvodnit potřebu vytvoření vlastních metod/nástrojů/algoritmů. Ve většině případů budete přebírat již navržené metody/algoritmy/definice/nástroje a je to pro školní projekt typické. Stejně tak je typické, že studenti chybně vymýšlí již hotové věci a dojdou k naprostému nesmyslu - je třeba toto nebezpečí eleminovat v tomto zdůvodnění.

Velmi důležité, až fanaticky povinné, kontrolované a hodnocené: na každém místě v textu, kde se poprvé objeví pojem z předmětu IMS bude v závorce uveden odkaz na předmět a číslo slajdu, na kterém je pojem definován. Pokud bude významný pojem z předmětu IMS takto nedokumentován v textu a zjevně bude používán nevhodným nebo nepřesným způsobem, bude tento fakt hodnocen bodovou ztrátou. Tento požadavek je míněn s naprostou vážností. Cílem je vyhnout se studentské tvůrčí činnosti ve vysvětlování známých pojmů, což mnohdy vede k naprostým bludům, ztrátě bodů a zápočtů. Pokud student pojem cituje korektně a přesto nekorektně používá, bude to hodnoceno dvojnásobnou bodovou ztrátou.

\subsection{Rozbor tématu}
Práce je zacílena na produkci, spotřebu a přepravu zemního plynu v zemích okolo
České republiky (Polsko, ČR, Slovensko, Maďarsko, Ukrajina) a proto bylo nejprve potřeba
nalézt tyto hodnoty pro každou z těchto zemí (\ref{Zeme}),
další stěžejní informací byla vzdálenost mezi jednotlivými zeměmi a také vlastnosti
plynovodů (rychlost toku, délka, maximální propustnost...)
mezi místem těžby zemního plynu a jednotlivými zeměmi (\ref{Preprava}).
Poslení důležitou informací je umístění a kapacita vznikajícíh terminálů na LNG,
které jsou v plánu v těchto zemích postavit případně, které by mohli danou zemi zásobovat
(\ref{lng}).

\subsubsection{Zemní plyn v jednotlivých zemích}\label{Zeme}
\begin{table}[h!]
\begin{center}
\begin{tabular}{|c|r|r|r|r|}
    \hline
    Stát 			& import 	& export & produkce & spotřeba \\
    \hline 
    Česká republika	& 8 479 		& 8 		& 252	& 8 477\\ 
    Slovensko 		& 5 579		& 15		& 124	& 5 820\\
    Maďarsko 		& 8 176		& 1 443	& 1 949	& 9 221\\
    Polsko 			& 12 473 	& 94 	& 6 206	& 18 229\\
    Ukrajina 		& 27 495 	& -		& 20 046 & 51 528 \\ \hline
\end{tabular}
\caption{Veškeré hodnoty jsou uvedeny v jednotkách $m^3 * 10^6$.
Zdroj \cite{IEA}}
\label{tabulka1}
\end{center}
\end{table}

\subsubsection{Přepravní trasy}\label{Preprava}

\subsubsection{Terminály LNG}\label{lng}

\section{Koncepce - modelářská témata}

Konceptuální model je abstrakce reality a redukce reality na soubor relevantních faktů pro sestavení simulačního modelu. Předpokládáme, že model bude obsahovat fakta z "Rozboru tématu". Pokud jsou některá vyřazena nebo modifikována, je nuto to zde zdůvodnit (například: zkoumaný subjekt vykazuje v jednom procentu případů toto chování, ovšem pro potřeby modelu je to naprosto marginální a smíme to zanedbat, neboť ...). Pokud některé partie reality zanedbáváte nebo zjednodušujete, musí to být zdůvodněno a v ideálním případě musí být prokázáno, že to neovlivní validitu modelu. Cílem konceptuálního (abstraktního) modelu je formalizovat relevantní fakta o modelovaném systému a jejich vazby. Podle koncept. modelu by měl být každý schopen sestavit simulační model.
kapitola 3.1: Způsob vyjádření konceptuálního modelu musí být zdůvodněn (na obrázku xxx je uvedeno schéma systému, v rovnicích xx-yy jsou popsány vazby mezi ..., způsob synchronizace procesů je na obrázku xxx s Petriho sítí).
kapitola 3.2: Formy konceptuálního modelu (důraz je kladen na srozumitelnost sdělení). Podle potřeby uveďte konkrétní relevantní:
obrázek/náčrt/schéma/mapa (možno čitelně rukou)
matematické rovnice - u některých témat (např. se spojitými prvky, optimalizace, ...) naprosto nezbytné. Dobré je chápat, že veličiny (fyzikální, technické, ekonomické) mají jednotky, bez kterých údaj nedává smysl.
stavový diagram (konečný automat) nebo Petriho síť - spíše na abstraktní úrovni. Petriho síť nemá zobrazovat výpočty a přílišné detaily. Pokud se pohodlně nevejde na obrazovku, je nepoužitelná. Možno rozdělit na bloky se zajímavými detaily a prezentovat odděleně abstraktní celek a podrobně specifikované bloky (hierarchický přístup).

\section{Koncepce - implementační témata}

Popište abstraktně architekturu vašeho programu, princip jeho činnosti, významné datové struktury a podobně. Smyslem této kapitoly je podat informaci o programu bez použití názvů tříd, funkcí a podobně. Tuto kapitolu by měl pochopit každý technik i bez informatického vzdělání. Vyjadřovacími prostředky jsou vývojové diagramy, schémata, vzorce, algoritmy v pseudokódu a podobně. Musí zde být vysvětlena nosná myšlenka vašeho přístupu.


\section{Architektura simulačního modelu/simulátoru}

Tato kapitola má různou důležitost pro různé typy zadání. U implementačních témat lze tady očekávat jádro dokumentace. Zde můžete využít zajímavého prvku ve vašem simulačním modelu a tady ho "prodat".
kapitola 4.1: Minimálně je nutno ukázat mapování abstraktního (koncept.) modelu do simulačního (resp. simulátoru). Např. které třídy odpovídají kterým procesům/veličinám a podobně.



\section{Podstata simulačních experimentů a jejich průběh}

Nezaměňujte pojmy testování a experimentování (důvod pro bodovou ztrátu)!!!
Zopakovat/shrnout co přesně chcete zjistit experimentováním a proč k tomu potřebujete model. Pokud experimentování nemá cíl, je celý projekt špatně. Je celkem přípustné u experimentu odhalit chybu v modelu, kterou na základě experimentu opravíte. Pokud se v některém experimentu nechová model podle očekávání, je nutné tento experiment důkladně prověřit a chování modelu zdůvodnit (je to součást simulačnické profese). Pokud model pro některé vstupy nemá důvěryhodné výsledky, je nutné to zdokumentovat. Pochopitelně model musí mít důvěryhodné výsledky pro většinu myslitelných vstupů.
kapitola 5.1: Naznačit postup experimentování – jakým způsobem hodláte prostřednictvím experimentů dojít ke svému cíli (v některých situacích je přípustné "to zkoušet tak dlouho až to vyjde", ale i ty musí mít nějaký organizovaný postup).
kapitola 5.2: Dokumentace jednotlivých experimentů - souhrn vstupních podmínek a podmínek běhu simulace, komentovaný výpis výsledků, závěr experimentu a plán pro další experiment (např. v experimentu 341. jsem nastavil vstup x na hodnotu X, která je typická pro ... a vstup y na Y, protože chci zjistit chování systému v prostředi ... Po skončení běhu simulace byly získány tyto výsledky ..., kde je nejzajímavější hodnota sledovaných veličin a,b,c které se chovaly podle předpokladu a veličin d,e,f které ne. Lze z toho usoudit, že v modelu není správně implementováno chování v podmínkách ... a proto v následujících experimentech budu vycházet z modifikovaného modelu verze ... Nebo výsledky ukazují, že systém v těchto podmínkách vykazuje značnou citlivost na parametr x ... a proto bude dobré v dalších experimentech přesně prověřit chování systému na parametr x v intervalu hodnot ... až ...)
kapitola 5.3: Závěry experimentů – bylo provedeno N experimentů v těchto situacích ... V průběhu experimentování byla odstraněna ... chyba v modelu. Z experimentů lze odvodit chování systémů s dostatečnou věrohodností a experimentální prověřování těchto ... situací již napřinese další výsledky, neboť ...


\section{Shrnutí simulačních experimentů a závěr}

Závěrem dokumentace se rozumí zhodnocení simulační studie a zhodnocení experimentů (např. Z výsledků experimentů vyplývá, že ... při předpokladu, že ... Validita modelu byla ověřena ... V rámci projektu vznikl nástroj ..., který vychází z ... a byl implementován v ...).
do závěru se nehodí psát poznámky osobního charakteru (např. práce na projektu mě bavila/nebavila, ...). Technická zpráva není osobní příběh autora.
absolutně nikoho nezajímá, kolik úsilí jste projektu věnovali, důležitá je pouze kvalita zpracování simulátoru/modelu a obsažnost simulační studie (rozhodně ne např.: projekt jsem dělal ... hodin, což je víc než zadání předpokládalo. Program má ... řádků kódu). Pokud zdůrazňujete, že jste práci dělali významně déle než se čekalo, pak tím pouze demonstrujete vlastní neschopnost (to platí zejména v profesním životě).
do závěru se velmi nehodí psát "auto-zhodnocení" kvality práce, to je výhradně na recenzentovi/hodnotiteli (např. v projektu jsem zcela splnil zadání a domnívám se, že můj model je bezchybný a výsledky taktéž). Statisticky častý je pravý opak autorova auto-zhodnocení. Pokud přesto chcete vyzdvihnout kvalitu svého díla (což je dobře), tak vaše výroky musí být naprosto nepopiratelně zdůvodněny a prokázány (např. pomocí jiného referenčního přístupu, matematického důkazu, analýzy, ...).


\newpage
\renewcommand{\refname}{Literatura}
\begin{thebibliography}{99}
\bibitem{IEA} IEA natural gas information [online]. [cit. 2014-12-03]. ISBN 1683-4267. Dostupné z: http://www.oecd-ilibrary.org/energy/natural-gas-information{\_}16834267
\bibitem{GIE} Gas Infrastructure Europe [online]. [cit. 2014-12-03]. Dostupné z: http://www.gie.eu.com/
\bibitem{wiki} Hypertext Transfer Protocol [online]. 2014 [cit. 2014-11-22].
Dostupné z: http://cs.wikipedia.org/wiki/Hypertext{\_}Transfer{\_}Protocol
\bibitem{pozadavky} Požadavky protokolu HTTP a jejich zpracování v PHP [online]. 2014 [cit. 2014-11-22].
Dostupné z: http://interval.cz/clanky/pozadavky-protokolu-http-a-jejich-zpracovani-v-php/
\end{thebibliography}
\end{document}
